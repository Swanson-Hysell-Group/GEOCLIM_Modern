\documentclass[11pt,letterpaper]{article}

%\usepackage{fontspec}
%\usepackage[utf8]{inputenc}
\usepackage{textcomp,marvosym}
\usepackage{amsmath,amssymb}
\usepackage[normalem]{ulem}
\usepackage[left]{lineno}
\usepackage{changepage}
\usepackage{rotating}
\usepackage{color}
\usepackage{natbib}
\usepackage{setspace}
\usepackage{}
\usepackage{fancyhdr}
\usepackage{graphicx}
\usepackage{xspace}
\usepackage{threeparttable}
\usepackage{color,colortbl}
\usepackage{url}
%\usepackage[hidelinks]{hyperref}
\urlstyle{same}
\doublespacing

\raggedright
\textwidth = 6.5 in
\textheight = 8.25 in
\oddsidemargin = 0.0 in
\evensidemargin = 0.0 in
\topmargin = 0.0 in
\headheight = 0.0 in
\headsep = 0.5 in
\parskip = 0.1 in
\parindent = 0.1in

% Bold the 'Figure #' in the caption and separate it from the title/caption with a period
% Captions will be left justified
\usepackage[aboveskip=1pt,labelfont=bf,labelsep=period,justification=raggedright,singlelinecheck=off]{caption}

% Remove brackets from numbering in List of References
%\makeatletter
%\renewcommand{\@biblabel}[1]{\quad#1.}
%\makeatother

% Self defined commands
\newcommand{\degC}{$^{\circ}$C\xspace}
\newcommand{\dC}{$\delta^{13}$C\xspace}
\newcommand{\dO}{$\delta^{18}$O\xspace}
\newcommand{\SrSr}{$^{87}$Sr/$^{86}$Sr\xspace}
\newcommand{\permil}{\textperthousand\xspace}
\newcommand{\dsil}{$d$\xspace}

\setcounter{figure}{0}
\renewcommand{\thefigure}{DR\arabic{figure}}
\setcounter{table}{0}
\renewcommand{\thetable}{DR\arabic{table}}

\definecolor{Yellow}{rgb}{1,1,0.35}
%

\pagestyle{myheadings}
\pagestyle{fancy}
\fancyhf{}
\lhead{Park et al., in review for XXX}
\rhead{\thepage}

\begin{document}

\begin{flushleft}
{\Large \textbf{Data Repository for ``GEOCLIM Modern''}}
\\
Yuem Park\textsuperscript{1},
Nicholas L. Swanson-Hysell\textsuperscript{1},
Pierre Maffre\textsuperscript{2},
Yves Godd\'eris\textsuperscript{2}
\\
\bigskip
\textsuperscript{1} Department of Earth and Planetary Science, University of California, Berkeley, CA, USA
\\
\bigskip

\end{flushleft}

\linenumbers

This document accompanies the discussion contained in the main text. All the Python code used for this study, as well as the associated data tables not included in this document, can be found at: \url{https://github.com/Swanson-Hysell-Group/XXX}.

\section*{Regolith Component}

The regolith component of GEOCLIM models the geochemical evolution of a rock particle as it leaves unweathered bedrock and transits through overlying regolith.

\begin{equation}
    \frac{dh}{dt} = P - E
\end{equation}

\begin{equation}
    \frac{\partial x}{\partial t} = -P \frac{\partial x}{\partial z} - K \tau^{\sigma}x
\end{equation}

\begin{equation}
    \frac{\partial \tau}{\partial t} = -P \frac{\partial \tau}{\partial z} + 1
\end{equation}

\clearpage

\singlespacing

\newpage

\bibliographystyle{gsabull}
\bibliography{References}

\end{document}
