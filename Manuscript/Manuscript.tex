\documentclass[11pt,letterpaper]{article}

%\usepackage{fontspec}
%\usepackage[utf8]{inputenc}
\usepackage{textcomp,marvosym}
\usepackage{amsmath,amssymb}
\usepackage[normalem]{ulem}
\usepackage[left]{lineno}
\usepackage{booktabs}
\usepackage{changepage}
\usepackage{rotating}
\usepackage{color}
\usepackage{natbib}
\usepackage{setspace}
\usepackage{array}
\usepackage{fancyhdr}
\usepackage{graphicx}
\usepackage{xspace}
\usepackage[hidelinks]{hyperref}
\urlstyle{same}
\usepackage{threeparttable}
\doublespacing

\raggedright
\textwidth = 6.5 in
\textheight = 8.25 in
\oddsidemargin = 0.0 in
\evensidemargin = 0.0 in
\topmargin = 0.0 in
\headheight = 0.0 in
\headsep = 0.5 in
\parskip = 0.1 in
\parindent = 0.2in

% Bold the 'Figure #' in the caption and separate it from the title/caption with a period
% Captions will be left justified
\usepackage[aboveskip=1pt,labelfont=bf,labelsep=period,justification=raggedright,singlelinecheck=off]{caption}

% Remove brackets from numbering in List of References
%\makeatletter
%\renewcommand{\@biblabel}[1]{\quad#1.}
%\makeatother

% Self defined commands
\newcommand{\degreesC}{\textdegree C\xspace}
\newcommand{\degrees}{\textdegree\xspace}
\newcommand{\dC}{$\delta^{13}$C\xspace}
\newcommand{\dO}{$\delta^{18}$O\xspace}
\newcommand{\SrSr}{$^{87}$Sr/$^{86}$Sr\xspace}
\newcommand{\permil}{\textperthousand\xspace}
\newcommand{\dsil}{$d$\xspace}
\newcommand{\UPb}{$^{206}$Pb/$^{238}$U\xspace}
\newcommand{\pCOtwo}{\textit{p}CO$_{2}$\xspace}
\newcommand{\COtwo}{CO$_{2}$\xspace}
%

\pagestyle{myheadings}
\pagestyle{fancy}
\fancyhf{}
\lhead{Park et al., in preparation for XXX}
\rhead{\thepage}

\begin{document}

\begin{flushleft}
{\Large \textbf{GEOCLIM Modern}}

Yuem Park\textsuperscript{1},
Nicholas L. Swanson-Hysell\textsuperscript{1},
Pierre Maffre\textsuperscript{2},
Francis A. Macdonald\textsuperscript{3},
Yves Godd\'eris\textsuperscript{2}

\bigskip
\textsuperscript{1} Department of Earth and Planetary Science, University of California, Berkeley, CA, USA

\textsuperscript{2} XXX
\bigskip

\end{flushleft}

\noindent\textit{This article is in preparation for XXX}

\linenumbers

\section*{ABSTRACT \label{sec:ABSTRACT}}

XXX

\section*{INTRODUCTION \label{sec:INTRODUCTION}}

The broader Indonesia archipelago has an out-sized influence on modern chemical weathering fluxes. The confluence of steep topography, a warm and wet tropical climate and the presence of mafic lithologies results in high fluxes of Ca and Mg cations and associated \COtwo consumption \citep{Hartmann2009a,Hartmann2014a}. There has been significant growth of land area within the region since the Miocene associated with ongoing arc-continent collision between the Sunda-Banda arc system \citep{Molnar2015a}.

Given that the region is a modern-day hotspot of chemical weathering and \COtwo consumption, what effect has this growth had on Earth's climate state and what would \pCOtwo on Earth be if the region wasn't there at all? Assessing the effect of the growth of the Indonesia region on Earth's \pCOtwo levels can not be determined simply through assessing the present-day chemical weathering fluxes. As \COtwo sinks are removed and \COtwo rises, increases in temperature and runoff cause other weathering sinks to increase. The steady-state \COtwo of Earth is set by the \pCOtwo level at which the sinks of \COtwo are equal to the sources. Based on paleogeographic reconstructions of arc-continent collisions through Earth history, \citet{Macdonald2019a} proposed that arc-continent collisions in the tropics are responsible for increasing planetary weatherability, lowering \pCOtwo and ind

XXXX

Over geologic time-scales, \COtwo enters Earth's ocean/atmosphere system primarily via volcanism and metamorphic degassing of the solid Earth, and leaves it primarily through the chemical weathering of silicate rocks, which delivers alkalinity and cations to the ocean that ultimately result in the consumption of carbon via the precipitation of carbonate rocks \citep{Kump2000a}. Imbalances between the magnitude of these source and sink fluxes will cause \pCOtwo to change until the silicate weathering negative feedback (in which elevated \pCOtwo leads to higher temperatures and a more intense hydrological cycle that enhances chemical weathering of silicate rocks and therefore the consumption of \COtwo, and vice versa) stabilizes \pCOtwo at a new steady state value \citep{Walker1981a}. One way to perturb steady state \pCOtwo is to change global weatherability - the sum of factors other than climate that contribute to the chemical weathering of rocks and associated \COtwo consumption, such as lithology \citep{Gaillardet1999a, Dessert2003a}, topography \citep{Maher2014a, Maffre2018a}, and paleolatitude \citep{Swanson-Hysell2017a}. \COtwo input from the solid Earth, assuming that it remains constant, can be removed via silicate weathering at a lower \pCOtwo on a planet with relatively high weatherability than on a planet where it is lower \citep{Kump1997a}.

Mafic rocks have a higher weatherability than felsic rocks because of both the higher concentrations of Ca and Mg (that ultimately sequester carbon through precipitation as carbonate) in mafic lithologies, as well as their faster weathering rates in identical conditions \citep{Dessert2001a, Dessert2003a}. Soil shielding is another important factor that affects weatherability - in low-relief regions, regolith tends to be thicker, which can lead to a supply-limited weathering regime in which the weathering rate of the underlying bedrock is dependent only on the uplift rate \citep{Gabet2009a, Maher2014a}. In this framework, processes that lead to continued exhumation of mafic lithologies and the creation of steep topography that minimizes soil shielding, particularly in tropical regions where temperature and runoff is high, should exert a strong control on global weatherability and long-term climate. Recent work has suggested that arc-continent collisions in the tropics during the Ordovician \citep{Swanson-Hysell2017a}, the Cenozoic \citep{Jagoutz2016a}, and even the Neoproterozoic \citep{Park2018a} played an important role in initiating glacial climate states at those times. Furthermore, a comparison between the paleolatitudinal position of all major Phanerozoic arc-continent collisions and the latitudinal extent of continental ice sheets reveals a strong correlation between the extent of glaciation and arc-continent collisions in the tropics \citep{Macdonald2019a}. On the other hand, \citet{Park2019a} found no correlation between the extent of glaciation and large igneous province area in the tropics, which was explained by the fact that large igneous provinces, although mafic, are often emplaced into low relief areas where soil shielding is relatively high.

However, quantifying the increase in global weatherability associated with these tropical arc-continent collisions or large igneous province emplacements and the consequent decrease in steady state \pCOtwo is challenging for the past, primarily due to the large uncertainties on the global context within which these events took place. For instance, in order to even begin to model global weathering fluxes and \pCOtwo, one would need to have constraints on the paleogeographic distribution of land masses, the spatial distribution of temperature, runoff, topography, and lithology on these land masses, and the carbon input flux coming from the degassing of the solid Earth at the time of these events. While the paleogeographic distribution of the continents is broadly well-constrained for recent Earth history (e.g. \citealp{Torsvik2016a}) and temperature and runoff can be estimated using a global climate model, constraints on the other parameters are poor, particularly as we go deeper into Earth's history.

However, the growth of the Malay Archipelago (the islands of Indonesia and surrounding regions, Fig. X) as Australia migrates northward represents an episode of arc-continent collision occurring in the tropics today. Given that the global context within which this event is occurring is well-constrained, the growth of the Malay Archipelago presents an opportunity to quantify the increase in global weatherability and consequent decrease in steady state \pCOtwo associated with it, and evaluate the feasibility of similar tropical arc-continent collisions initiating glacial climate states throughout Earth's history \citep{Macdonald2019a}.

\section*{METHODS}

\subsection*{GEOCLIM}

To estimate the decrease in steady state \pCOtwo associated with the increase in global weatherability from the growth of the Malay Archipelago, we use the spatially resolved GEOCLIM Earth system model \citep{Godderis2014a, Godderis2017c}, which iteratively couples box models of surface biogeochemical cycles to climate model output. Estimates of fluxes of species between carbon, alkalinity, phosphorous, and oxygen reservoirs in the ocean and atmosphere are used to calculate \pCOtwo, which is then compared against climate model output computed at various \pCOtwo levels to estimate temperature and runoff at the current \pCOtwo in GEOCLIM. These new temperature and runoff fields are then fed back into the biogeochemical box models to update \pCOtwo, for which new temperature and runoff fields are again estimated. This iteration is continued until a steady state is achieved in all reservoirs.

In this study, we use temperature and runoff fields from a subset of the GFDL CM2.0 experiments \citep{Delworth2006a, Delworth2006b} for the climate model component of GEOCLIM. These experiments were performed in order to explore the effect of various changes in forcing agents on climate since ca. 1860 at a 1\degrees $\times$ 1\degrees resolution. In the `1860 control' experiment, forcing agents representative of conditions ca. 1860 (including \COtwo, CH$_{4}$, N$_{2}$O, O$_{3}$, sulfates, carbon, dust, sea salt, solar irradiance, and the distribution of land cover types) are held constant for 500 years after reaching equilibrium. \pCOtwo in 1860 is assumed to be 286~ppm. In the `+1\%/yr to 2$\times$' experiment, initial conditions are taken from the `1860 control' experiment, then \pCOtwo is prescribed to increase from 286~ppm at a compounded rate of +1\% per year for 70 years, when \pCOtwo reaches double (572~ppm) of the initial value. \pCOtwo is then held constant until the end of the 280 year experiment. All non-\COtwo forcing agents are held constant. The `+1\%/yr to 4$\times$' experiment is identical to the `+1\%/yr to 2$\times$' experiment, except that \pCOtwo is prescribed to increase for 140 years, when \pCOtwo reaches quadruple (1144~ppm) of the initial value. We take the mean of the last 100 years of each of these three experiments (when \pCOtwo is being held constant at its final level) to obtain temperature and runoff fields for 286, 572, and 1144~ppm respectively. The primary strength of using these three GFDL CM2.0 experiments is that all non-\COtwo forcing agents are held constant, allowing the effect of \pCOtwo on climate to be isolated.

The silicate weathering component of GEOCLIM calculates \COtwo consumption by silicate weathering for each continental grid cell. In previous versions of the model, the silicate weathering rate was a function of temperature and runoff only, and assumed that the bedrock in all continental grid cells had identical chemical compositions (e.g. \citealp{Godderis2014a}). However, more recent versions of GEOCLIM implement regolith development and soil shielding into the silicate weathering component based on work by \citet{Heimsath1997a}, \citet{Gabet2009a}, \citet{West2012a}, \citet{Carretier2014a}, \citet{Godderis2017b}, and \citet{Maffre2018a}, which introduces an additional dependence on topographic slope (see Data Repository). While this introduction of regolith development into GEOCLIM is an important piece of assessing the impact of arc-continent collisions in the Malay Archipelago on \pCOtwo, the relatively high Ca and Mg concentration in arc rocks relative to other lithologies must also be considered. We therefore implement variable bedrock Ca+Mg concentration into the silicate weathering component of GEOCLIM. The spatial distribution of lithologies is sourced from the global lithologic map (GLiM) of \citet{Hartmann2012a}. 16 lithologic categories are represented in GLiM, which we simplify into 6 broader categories: metamorphic, felsic, intermediate, mafic, carbonate, and siliciclastic sediment (Fig. XX; see Data Repository). Each continental grid cell is then assigned one of these lithologic categories. The Ca+Mg concentrations of felsic, intermediate, and mafic grid cells are then assigned based on the mean of MgO and CaO measurements on rocks of each of these lithologic categories in the EarthChem Portal (\url{www.earthchem.org/portal}). The weathering of carbonate grid cells does not contribute to long-term \COtwo consumption and therefore its Ca+Mg concentration is ignored. The Ca+Mg concentrations of metamorphic and siliciclastic sediment grid cells are more difficult to define, since the chemical composition of these two lithologic classes is strongly dependent on its protolith. The chemical composition of siliciclastic sediment is further strongly dependent on its degree of chemical depletion. We therefore explore a range of feasible Ca+Mg concentrations of metamorphic and siliciclastic sediment grid cells during the calibration of the silicate weathering component of GEOCLIM (see below).

\subsection*{Parameter Calibration}

The silicate weathering component of GEOCLIM models the geochemical evolution of a rock particle as it leaves unweathered bedrock and transits through overlying regolith (see Data Repository). As the rock particle transits through the regolith, some fraction of the weatherable phases in the rock particle are dissolved and leaves the regolith column as weathered phases. The fraction of the weatherable phases in the rock particle that are not dissolved after transiting through the regolith column leaves it via erosion. Central to this model is the dissolution rate constant, which describes the rate at which weatherable phases in the rock particle are dissolved as it transits through the regolith:

\begin{equation}
    k_{d}\left(1-e^{-k_{w}q}\right)e^{-\frac{E_{a}}{R}\left(\frac{1}{T}-\frac{1}{T_{0}}\right)}\tau^{\sigma}
    \label{eq:1}
\end{equation}

$q$ is the runoff, $E_{a}$ is the activation energy, $R$ is the universal gas constant, $T$ is the temperature, $\tau$ is the time that a given rock particle has spent in the regolith, and $k_{d}$, $k_{w}$, and $\sigma$ are calibration constants.



clean up equation - add units?

allbcc vs lith map

XXX


\begin{itemize}
    \item GEOCLIM and GFDL models
    \item added lithology (GLiM)
    \item calibration
    \item tested scenarios
    \item in discussion, talk about GFDL limitations (equilibrated by last 100 years?, not changing paleogeography, low resolution, model is quite old)
\end{itemize}

The global lithological map database GLiM of Hartmann and Moosdorf was used to develop a

The lithological categories of the global lithological map database GLiM of \cite{Hartmann2012a} were used and grouped into 6 categories: metamorphic, felsic, intermediate, mafic, carbonate, and siliciclastic. These values were assigned at the 0.5\textdegree resolution and were then downsampled using a mode downsampling scheme to 1\textdegree resolution

\section*{RESULTS}

XXX

\section*{DISCUSSION}

XXX

\section*{ACKNOWLEDGEMENTS \label{sec:ACKNOWLEDGEMENTS}}

XXX

\clearpage
\newpage

\section*{TABLES}

XXX

\clearpage
\newpage

\section*{FIGURES}

\begin{itemize}
    \item lithologic mask
\end{itemize}

\clearpage
\newpage
\footnotesize

\singlespacing

\bibliographystyle{gsabull}
\bibliography{References}

\end{document}
